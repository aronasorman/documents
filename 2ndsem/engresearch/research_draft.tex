\documentclass[12pt,letterpaper]{article}
\usepackage{ifpdf}
\usepackage{mla}
\usepackage{outlines}
\def\thesubsubsection{}
\begin{document}
\begin{mla}{Aron Fyodor M.}{Asor}{En12 R40 / 090275}{Ms. Sandra Nicole Roldan}{12 February 2010}{Protecting male rape victims here and abroad}

\subsubsection{Male rape is real}
%
% insert facts and figures to convince readers that male rape is happening
%
\tab Rape is a broad and serious issue globally. According to the latest figures of the Philippine Government's National Statistics Office our country alone had 1,267 reported rape cases from 2008 or an average of almost 4 rape cases per day (21). But according to Graham male rape is still a predominantly female issue; he states that the public only recognizes females as the victims of rape and disregards males, which results in the lack of statistics and general information on male rape specifically (187).

%
% define rape and male rape, clarify some issues regarding these two definitions i.e. homosexuality etc.
% (try to transition to psychology)
%
It is important at this early stage to define some terms that will be frequently used to avoid confusion. \textit{Rape} in context of this paper is the forced, physical attempt at sexual gratification by a perpetrator to an unwilling victim. \textit{Male rape} therefore is rape wherein the intended victim is a male. It is important to emphasize that homosexual males can  be raped using the definition given. People often stereotype homosexual men, thinking that most of them entertain the idea of male-on-male sex and would thus consent or even pursue such acts. But according to the findings of Groth and Burgess it is often heterosexual men who commit male rape and in an analysis of 22 cases of male rape in a community setting, the gender of the victim did not appear to be of primary importance to some of the rapists, but for others, males appeared to be specific intended targets, and that accessibility to the target is a more important factor than age, appearance and race (806).

%
% detail what the male rape victim undergoes after being raped, say some things about male dominated society etc.
%
The preference for male victims by the rapists in the aforementioned study presents a conflicting view of what the general public traditionally has in mind for rape primarily as acts of forced sexual pleasure. Scarce pointed out that male rape would be better understood if seen as an act of power instead of sexual gratification (par. 5). He cites one example in US prisons wherein males who are ``at the top of the pack'' would rape other males, preferably known homosexuals and would call them derogatory names like ``bitch'' and ``girl.''(par. 9) Schwartz and DeKeseredy expound on this view by saying that people are raised in a patriarchy, or specifically a \textit{societal patriarchy}, which among other things emphasizes on male superiority and domination against females and other males (61-74). This type of society turns males into rapists by teaching them to objectify both their fellow males and females and treating them as simply commodities and subhumans which must be beaten down if seen as superior to themselves. This creates a society that encourages the development of egocentric males who support each other and further encourage sexual behavior and the notion of ego and pride as the center of a man's life and driving force for his actions (Schwartz and DeKeseredy 68-88). Rape therefore can be seen as the ultimate device for male dominance and for ``bringing people down to their right place.'' It boosts the ego of the male perpetrator while severely degrading the male victim's. The victim loses self-esteem for being ``reduced to a woman's position'' and ``being someone else's bitch.'' According to Sable et al. being raped reduces a male's masculine self-identity and ego, and due to this the male victims lose the courage to file a complaint and would rather not report to the police at all, or if they would, report it to a female police officer instead of another male for fear of ``admitting to a guy\ldots how much they failed as a man'' (497). This is supported by the survey conducted by Davies et al. wherein out of the 40 male rape victims interviewed only 5 have ever reported to the police (qtd. in Jamel et al. 534).
%
% discuss about lack of misinformation aggravates the situation.
%
This barrier for male rape victims is further aggravated by the lack of information by the general public. According to Scarce ``discussions about male rape are frequently absent\ldots because the general public and popular culture have traditionally viewed rape in context of violence against women'' (par. 8). He states that while there are numerous information campaigns and sex education curriculums centered around females , males as the victim of rape is still largely unheard of or otherwise sneered at by the general public due to the lack of awareness of this kind of rape (par. 11). Sable says that one of the barriers for reporting male rape cases is general distrust with the police and ``unawareness on how to obtain help.'' (5). The male rape victim is left clueless on how to go about and react: should he report immediately to the police station? Is there a hotline specifically for male rape victims? Are there counselors that he can talk to and cater especially to his needs? What laws can he utilize?

\subsubsection{Protection under the law}
%
% discuss and summarize every part of ra 8353 with regards to protection for male rape victims
%
\tab Unfortunately only a few countries and states have a similar definition of rape as one given by this paper. It may be argued that the definition given by this paper is vague and ambiguous, but it seeks to cover all kinds of rape wherein the female is not the only victim. This is opposite to the goals of a law. More often than not, a law explicitly gives a definition of rape and gives a scenario of what can be classified as rape in the country or state to avoid confusion during court trials. Unfortunately this attempt at an explicit and clear definition to rape leaves out the ``corner cases'' or those not foreseen or recognized by the lawmaker(s) as legitimate forms of rape. The lawmaker(s) may also be culturally biased in the formation of such definition. 

Here in the Philippines the crime of rape is covered by Republic act no. 8353, also known as the Anti-rape law of 1997. It was an amendment to the Revised Penal Code of 1932 that was meant to address the growing concern of underage sex and paedophilia. The following is the definition as put forward by the law:

\begin{mlaquote}
	Article 266-A. \textit{Rape; When And How Committed}. - Rape is Committed-
	\begin{outline}[enumerate]
			\1 By a man who shall have carnal knowledge of a woman under any of the following circumstances:
				\2 Through force, threat or intimidation;
				\2 When the offended party is deprived of reason or otherwise unconscious;
				\2 By means of fraudulent machination or grave abuse of authority; and
				\2 When the offended party is under twelve (12) years of age or is demented, even though none of the circumstances mentioned above are present.
			\1 By any person who, under any of the circumstances mentioned in paragraph 1 hereof, shall commit an act of sexual assault by inserting his penis into another person's mouth or anal orifice, or any instrument or object, into the genital or anal orifice of another person.
	\end{outline}
\end{mlaquote}

Here the definition is very clear on what constitutes a rape. Rape is committed when the victim is forced to have sex or when the victim is unconscious or cannot think clearly. Rape is also committed automatically when the victim is less than twelve years of age, whether the victim has given her consent or not.

What the law is also clear about though, is that rape can only be committed upon a woman by a man; in paragraph 1 the law used the terms man and woman to denote the perpetrator and victim respectively. Paragraph 2 taken alone may encompass male rape; unfortunately it is dependent upon the circumstances presented by paragraph 1 which needs a man lusting for a woman for rape to be committed. Cabahug reaffirms that this definition can only be applied to female victims and that ``under Philippine laws, rape cannot be committed against male persons.''

This presents a huge problem for the Filipino male rape victim where even the law cannot protect him fully. The country itself does not legally recognize the phenomenon of male rape, and we could also expect that our country has no support system like hotlines and counseling that are meant to cater to this kind of rape. Thus if the male rape victim still wants to pursue his case he has to apply for other laws. According to Cabahug he can apply for the Crime of Lasciviousness under Article 336 of the Revised Penal Code or if he has sustained injuries, the Crime of Physical Injuries and so on depending on the condition of the male rape victim after being raped.

%
% compare to scotland rape law
%
Scotland does not suffer from this kind of legal shortsightedness. Scotland's Sexual Offences Act of 2009 spans 64 pages covering a wider range of topics than our own law. Aside from defining rape similar to paragraph 2 of our own rape law, Scotland's Act takes into account the more esoteric kinds of rape like raping corpses and animals and tackles the issue of voyeurism, has separate sections regarding children of different age brackets and defines the crime of rape, sexual assault and penetration instead of simply the crime of rape. Most importantly, male rape is taken into account through the definitions put forth by the crime of sexual assault by penetration. It would be beneficial for the Philippines to take this law from Scotland as inspiration for the next revision or our anti-rape laws.

%
% detail what support systems (or the lack thereof) are setup here in the Philippines
%
\subsubsection{Supporting male rape victims here in the Philippines}
Due to the lack of support from the legal system itself, it is no wonder why there are little to no support systems for male rape victims here in the Philippines. A quick search at the hotline page provided by the PLDT Yellow Pages dated 2007 shows no specific hotline for male rape victims but one for female victims named the ``Woman's Help Desk'' (cover).Consequently there are no rape crisis centers setup for male rape victims; as put forth by the Rape Victim Assistance Law of 1998, only women's desks in police stations and rape crisis centers for rape victims as recognized by our law need to be setup; consequently police officers are trained only to service female rape complainants (2-4). This further deters the male rape victim from reporting his case knowing that the support systems set up cater specifically to females.

%
% conclusion na yey! make a summary of weak points and cite recommendations by authors
%
\subsubsection{Recommendations}

\tab The Philippine Government has to remedy these shortcomings in the legal and support system and close the gap between the protection they offer between male and female rape victims. Though a minority of the projected overall number of rape cases every year, male rape is the result of a number of issues prevailing in society. The foremost issue is the societal patriarchy put forward by Schwartz and DeKeseredy that most of the males and females grow up with; subtle messages and actions in society that puts males' egos into the center stage (59-94). Resolving this issue by changing media messages and the general public's perceptions and psychology would take generations but the crimes of both male and female rape, abuse and all other male-centered crimes would decrease as a result (Schwartz and DeKeseredy 139-172). A shorter remedy to the current abysmal situation of male rape for our country would be to broaden the laws to encompass not only male rape but other more esoteric kinds of rape and clearly define the definitions of each rape case, similar to Scotland's Sexual Offences Act. Sable et al. recommends an information campaign to educate the general public about this kind of rape and how they can react to it in the case that they are the victims themselves or they know someone who has been victimized by this kind of rape (4). Jamel et al. recommends the retraining of police officers to cater to the special needs of male rape survivors (17). Scarce recommends that governments prepare resources and materials for male rape victims before a victim contacts the government, instead of procuring it after, increasing responsiveness to the situation and the confidence of the male rape victim to the system (par. 15). Scarce also recommends the creation of culturally competent sex education programs in campuses to educate students on male rape and clarify some myths regarding male sexuality (par 20).







\begin{workscited}
	
\bibent
Davies, Michelle, Paul Rogers and Jo-Anne Bates.``Blame towards male rape victims in a hypothetical sexual assault as a function of victim sexuality and degree of resistance.'' \textit{Journal of Homosexuality} 55.3 (2008): 533-544. Print.

\bibent
Ellis, Lee. Theories of Rape: \textit{Inquiries into the Causes of Sexual Aggression}. North Dakota: Hemisphere, 1989. Print.

\bibent
Graham, Ruth.``Male rape and the careful construction of the male victim.'' \textit{Social and Legal Issues} 15.2 (2006): 187-208. \textit{Academic Source Complete}. EBSCO. Web. 29 Jan. 2010.

\bibent
Philippines. National Statistics Office. \textit{Philippines in figures 2009}. NSO, 2009. 4 Feb. 2010.

\bibent
Sable, Marjorie, et al.``Barriers to reporting sexual assault for women and men.'' \textit{Journal of American College Health} 55.3 (2006): 157-162. Print.

\bibent
Scarce, Michael.``Same sex rape of male college students.'' \textit{Journal of American College Health} 45.4 (1997): 171. \textit{Biomedical Reference Collection: Basic}. EBSCO. Web. 16 Jan. 2010.

\bibent
Schwartz, Martin and Walter DeKeseredy. \textit{Sexual Assault on the College Campus: The Role of Male Peer Support}. California: SAGE, 1997. Print.

\end{workscited}


\end{mla}
\end{document}

Copyright (c) 2010 Aron Fyodor M. Asor

Permission is granted to copy, distribute and/or modify this document under the terms of the GNU Free Documentation License, Version 1.2 or any later version published by the Free Software Foundation; with no Invariant Sections, no Front-Cover Texts, and no Back-Cover Texts. A copy of the license is included in the section entitled ``GNU Free Documentation License.''

