
\section{Interview Details}
	\subsection{Boen Dorotheo R. Cabahug}	
	\begin{outline}
		\1 Lawyer 
		\1Roll Number 45683
		\1 bdrcabahug@yahoo.com, 09285053378
		\1 January 21, 2010, through electronic mail.
		\1 Questions:
			\2 Currently the Philippines depends largely on its immigrants and migrant workers  to financially support it during this recession. Do we have any laws that can protect these migrants in case they are abused in the countries they reside in?
			\3 If we do, what are these laws? Are they currently being implemented?
			\3 If not, what certain measures has the Government taken to ensure the safety of these people?
			\2 Republic Act 8353 also known as the Anti-rape law of 1997,chapter 3 article 266-A paragraph 1 includes a definition of rape which says ``rape is committed by a man who shall have carnal knowledge of a woman\ldots by inserting his penis into the genital or anal orifice of another person''. do you think this definition is up-to-date in this age and time?
			\3 Can this law be applied to other types of rape like male-to-male or female-to-male rape wherein the the male is the victim?
			\3 Senator Angara passed a bill last August 2009 that hopes to amend RA 8353. Does this new bill introduce measures that protect male rape victims?
			\2 What other laws can a male rape victim use here in the Philippines to protect himself against his offenders?
			\3 Can these laws be used to help Filipino male rape victims abroad?
	\end{outline}
