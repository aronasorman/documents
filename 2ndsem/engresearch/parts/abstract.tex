My research intends to show the protection that male migrants can expect from the Philippines' and their presiding country's legal system. My main focus is the different laws that a male rape victim in another country (for example, Scotland) can utilize to protect himself and to see if he can seek protection from the Philippines. My tentative thesis is:\textit{ The Philippine Government is currently ignoring these Filipino male rape victims and must give them the protection and attention that their female counterparts enjoy}. An integral factor in the beginning of my research are facts and figures showing that male rape is real and a significant threat to males in any society. I first have to show the reader that this type of rape is indeed happening everywhere around us. My next step will be to show that both male and female rape victims undergo the same trauma and experience. I will detail the psychology that female rape victims   undergo and extend this to males. I will then elaborate the reasons why males don't report this kind of crimes and why often they choose to be silent, even to their family and peers, and why one of these reasons is the bias of rape laws. I will then show the inadequacy of our current laws against male rape, its bias for female victims and compare these laws against other foreign laws like the current Sexual Offences Act in Scotland, which is by far the most adequate anti-rape law I have seen so far. If I see that the Philippine law is fairly adequate globally, I will find out if it is possible for Filipino male rape victims abroad to seek help from the Philippine Government in any way possible. Lastly, I will summarize my points, illustrate the weak spots in protecting male rape victims abroad and urge the Government to step up and protect these victims.

