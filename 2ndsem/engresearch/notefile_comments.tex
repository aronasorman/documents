\documentclass[12pt,letterpaper]{article}
\usepackage{ifpdf}
\usepackage{mla}
\begin{document}
\begin{mla}{Aron Fyodor M.}{Asor}{Mrs. Sandra Nicole Roldan}{En 12 R-40}{\today}{Notefile comments}

	Our own \textbf{anti-rape law} is seriously lacking and limited, compared to foreign laws like in Scotland.
	It's only limited to female and adolescent victims. For me this reflects the general attitude we have towards
	sexuality: that women are the ones that can only be raped, and men are automatically relegated to being the
	perpetrators and rapists, and implicitly states that there are no homosexuals who are raping and getting
	raped in our society. Ugh.

	This study by \textbf{Sable et. al.} can and will bring support to one of my points, that male rape victims 
	don't report mainly due to their pride and how their peers will view them. This is in line with what 
	\textbf{Schwartz and DeKeseredy} said that in the patriarchy we live in, men value their pride in the highest
	regard and will do anything to lift themselves higher than their peers. Thus an occassion such as rape would
	be damage their pride; men would do anything to prevent something like that. (I sound like a hypocrite, being
	male and all)

	This brings me to the study by \textbf{Rogers et. al.} He states that males are indeed biased on their 
	opinions on male rape unlike women. It is not unreasonable to assume that peers of a male victim would view
	said victim upon a different light. (which is often not a positive one)

	This \textbf{Interview} really worked for me. Every answer agrees with my thesis! Yey! And not only that,
	but it also introduces new leads, like searching for programs in the DFA that help OFWs. 

	This work by \textbf{Michael Scarce} wherein he makes recommendations for protecting male victims is good and
	would be inline with the kind of conclusion I am planning. It's important to note that Scarce is himself a 
	victim of rape. And he's male too.
	

\end{mla}
\end{document}
