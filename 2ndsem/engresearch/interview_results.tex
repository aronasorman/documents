\documentclass[12pt,letterpaper]{article}
\usepackage{ifpdf}
\usepackage{mla}
\usepackage{outlines}
\begin{document}
\begin{mla}{Aron Fyodor M.}{Asor}{Mrs. Sandra Nicole Roldan}{En 12 - R40 / 090275}{\today}{.}



	\subsection{Boen Dorotheo R. Cabahug}	
	\begin{outline}
		\1 Lawyer 
		\1Roll Number 45683
		\1 bdrcabahug@yahoo.com, 09285053378
		\1 January 21, 2010, through electronic mail.
		\1 Questions:
			\2 Currently the Philippines depends largely on its immigrants and migrant workers  to financially support it during this recession. Do we have any laws that can protect these migrants in case they are abused in the countries they reside in?
			\3 If we do, what are these laws? Are they currently being implemented?
			\3 If not, what certain measures has the Government taken to ensure the safety of these people?
			\2 Republic Act 8353 also known as the Anti-rape law of 1997,chapter 3 article 266-A paragraph 1 includes a definition of rape which says ``rape is committed by a man who shall have carnal knowledge of a woman\ldots by inserting his penis into the genital or anal orifice of another person''. do you think this definition is up-to-date in this age and time?
			\3 Can this law be applied to other types of rape like male-to-male or female-to-male rape wherein the the male is the victim?
			\3 Senator Angara passed a bill last August 2009 that hopes to amend RA 8353. Does this new bill introduce measures that protect male rape victims?
			\2 What other laws can a male rape victim use here in the Philippines to protect himself against his offenders?
			\3 Can these laws be used to help Filipino male rape victims abroad?
	\end{outline}


\subsection{Answers}
\begin{outline}[enumerate]
	\1  In case of abuse abroad,our brother Filipinos can take refuge of the criminal laws of the foreign country which is more or less similar to our own.  This is because the situs of the crime will determine the jurisdiction of the courts   thus, if the crime happens abroad it can only be tried therein.This is subject however to universal crimes such as genocide and terrorism which can be tried in another country other than where it is committed. 
		\2 It is my humble submission that our domestic laws cannot protect our OFWs if the crime is committed against them in another country.But if the abuse happens here, we have several laws that can be applied regardless of whether the victim is a Filipino or not.
		\2 In case of abuse committed abroad, our Government through the DFA has certain programs to assist the Filipino-victims.   
	\1 The definition of Rape is the latest revision made by our Congress taken into account the modern and prevailing situation. Thus, the previous understanding of rape as having carnal knowledge must give way to the existing definition. In this aspect, it is appropriate in the present time.
	\1  It is submitted that the current definition of rape can only be applied to female victims under the various ways enumerated under R.A. 8353. Under our Philippine laws, rape cannot be committed against male persons.
	\1 I'm sorry but I'm not aware of the Angara Bill so I cannot comment on that yet.
	\1 Since there can be no rape on male victims, the least he can use of is the crime of acts of lasciviousness under Article 336 of the Revised Penal Code. If there are injuries committed on him, he can also use the crime of physical injuries. And if he dies, homicide or murder as the case maybe.
	\1 Based on my understanding above-mentioned, these domestic laws cannot be applied if the crime is committed abroad. 
\end{outline}
\end{mla}
\end{document}
