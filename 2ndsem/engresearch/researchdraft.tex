\documentclass[letterpaper]{article}

\usepackage{outlines}
\usepackage{ifpdf}
\usepackage{mla}
\usepackage[american]{babel}
\usepackage{csquotes}
\usepackage[style=mla,hyperref=false]{biblatex}
\bibliography{biblio}

\def\thepart{\Alph{part}}
\def\thesection{\Alph{section}.}
\def\thesubsection{\thesection\normalsize{\arabic{subsection}}}
\begin{document}
\begin{mla}{Aron Fyodor M.}{Asor}{090275 / 1 BS Management}{En12 - R40 / Ms. Sandra Nicole Roldan}{20 January 2010}{. }


\section{Research Topic}

	\noindent Group General Topic: Migration

	\noindent  Narrowed Topic: The Legal issues regarding the sexual abuse of Filipino male migrants in the countries in which they reside in.

	\noindent Tentative Stand for Argument: It is high time the Philippine Government took up measures to protect the male migrants and OFWs the same way they protect their female counterparts.


	\begin{outline}[itemize]
	 \1 Keyword Definitions:
	 \2 Male Peer Support Model Theory- A theory which states that male groups like fraternities and brotherhoods support and encourage certain traits that usually undermine, degrade and objectify the people around them.

	 \2 Migration- In context of this research is the temporary or permanent movement of Filipinos to another country for work or residence.
	
	 \2 Rape- Is the forceful attempt at sexual gratification wherein one of the individuals involved refuses to become sexually intimate.
	\1 Backgound Questions:

	\2 What makes a certain male vulnerable to sexual abuse?

	\2 What hinders a male from reporting sexual abuse he has experienced?

	\2 How do the victims and their families cope with abuse (if he chooses to tell it to them)?

	\1 Secondary Questions:	
	\2 Are there laws in other countries made especialy to protect these sexually abused male migrants?

	\2 What are the legal course of actions these migrants may take?

	\2 Can they seek legal help from the Philippine Government?
	

	\end{outline}
\section{Interview Details}
	\subsection{Boen Dorotheo R. Cabahug}	
	\begin{outline}
		\1 Lawyer 
		\1Roll Number 45683
		\1 bdrcabahug@yahoo.com, 09285053378
		\1 January 21, 2010, through electronic mail.
		\1 Questions:
			\2 Currently the Philippines depends largely on its immigrants and migrant workers  to financially support it during this recession. Do we have any laws that can protect these migrants in case they are abused in the countries they reside in?
			\3 If we do, what are these laws? Are they currently being implemented?
			\3 If not, what certain measures has the Government taken to ensure the safety of these people?
			\2 Republic Act 8353 also known as the Anti-rape law of 1997,chapter 3 article 266-A paragraph 1 includes a definition of rape which says ``rape is committed by a man who shall have carnal knowledge of a woman\ldots by inserting his penis into the genital or anal orifice of another person''. do you think this definition is up-to-date in this age and time?
			\3 Can this law be applied to other types of rape like male-to-male or female-to-male rape wherein the the male is the victim?
			\3 Senator Angara passed a bill last August 2009 that hopes to amend RA 8353. Does this new bill introduce measures that protect male rape victims?
			\2 What other laws can a male rape victim use here in the Philippines to protect himself against his offenders?
			\3 Can these laws be used to help Filipino male rape victims abroad?
	\end{outline}
	\subsection{Fermin B. Nasol}
	\begin{outline}
		\1 National Bureau of Investigation
		\1 09278705223
		\1 (Interview pending)
		\1 Questions:
			\2 Rape is traditionally associated with females being the victims. But there are also male victims of rape happening everyday. How often does this type of rape get reported yearly to the NBI? What was the ratio of male to female victims?
			\3 Was there an increase in the amount of male rape victims over time?
			\3 On average, how many male rape victims get reported everyday?
			\4 Do you have an estimate of how many male rape cases go unreported? If so, how many?
	\end{outline}

\section{Abstract}


My research intends to show the protection that male migrants can expect from the Philippines' and their presiding country's legal system. My main focus is the different laws that a male rape victim in another country (for example, Scotland) can utilize to protect himself and to see if he can seek protection from the Philippines. My tentative thesis is:\textit{ The Philippine Government is currently ignoring these Filipino male rape victims and must give them the protection and attention that their female counterparts enjoy}. An integral factor in the beginning of my research are facts and figures showing that male rape is real and a significant threat to males in any society. I first have to show the reader that this type of rape is indeed happening everywhere around us. My next step will be to show that both male and female rape victims undergo the same trauma and experience. I will detail the psychology that female rape victims   undergo and extend this to males. I will then elaborate the reasons why males don't report this kind of crimes and why often they choose to be silent, even to their family and peers, and why one of these reasons is the bias of rape laws. I will then show the inadequacy of our current laws against male rape, its bias for female victims and compare these laws against other foreign laws like the current Sexual Offences Act in Scotland, which is by far the most adequate anti-rape law I have seen so far. If I see that the Philippine law is fairly adequate globally, I will find out if it is possible for Filipino male rape victims abroad to seek help from the Philippine Government in any way possible. Lastly, I will summarize my points, illustrate the weak spots in protecting male rape victims abroad and urge the Government to step up and protect these victims.

\section{Thesis and Outline}
Tentative Thesis:\textit{ The Philippine Government is currently ignoring these Filipino male rape victims abroad and must give them the protection and attention that their female counterparts enjoy.}
\begin{outline}[enumerate]
	\1 Introduction
	\2 Male rape facts and figures
	\2 Male rape victims experience the same trauma and psychology as females victims do
	\3 How the individual reacts
	\3 How the family and friends react
	\3 Why males don't report to the police
	\1 Body
	\2 Analysis of the laws related to sexual abuse and rape here in the Philippines
	\3 What protection these laws can offer
	\3 How they are limited in protecting male rape victims
	\3 How these laws fare against foreign anti-rape laws
	\2 Seeking help from the Philippine Government
	\3 Why seek help from the home country
	\3 What kind of protection can the Philippine Government offer to migrants
	\1 Conclusion
	\2 What legal measures can the Philippine Government offer for Filipino male rape victims here and abroad
	\2 Which foreign countries offer the best and worst protection for male rape victims
\end{outline}

\newpage
\printbibliography

\end{mla}

\end{document}
