\documentclass{slides}
\begin{document}
\title{Poetry Analysis: Order for Masks by Virginia R. Moreno}
\author{Aron Fyodor M. Asor\\
	Lit 13 R40\\
	090275}
\maketitle
	To this harlequinade

	I wear black tight and fool's cap

	Billiken*, make me three bright masks

	For the three tasks in my life.

	Three faces to wear

	One after the other

	For the three men in my life.

	When my Brother comes

	make me one opposite

	If he is a devil, a saint

	With a staff to his fork

	And for his horns, a crown.

	I hope for my contrast

	To make nil

	Our old resemblance to each other

	and my twin will walk me out

	Without a frown

	Pretending I am another.

	When my Father comes

	Make me one so like

	His child once eating his white bread in trance

	Philomela* before she was raped. I hope by likeness

	To make him believe this is the same kind

	The chaste face he made,

	And my blind Lear* will walk me out

	Without a word

	Fearing to peer behind.

	If my lover comes,

	Yes, when Seducer comes

	Make for me the face

	That will in color race

	The carnival stars

	And change in shape

	Under his grasping hands.

	Make it bloody

	When he needs it white

	Make it wicked in the dark

	Let him find no old mark

	Make it stone to his suave touch

	This magician will walk me out

	Newly loved.

	Not knowing why my tantalizing face

	Is strangely like the mangled parts of a face

	He once wiped out.

	Make me three masks.

	\newpage

	\[Poetry Analysis\]

	The title of the poem is Order for Masks.

	It was written by Virginia R. Moreno.

	This poem is written in the 1st person point of view.

	The speaker is someone who was unsatisfied with the image she projects to the three most important men in her life.

	The speaker puts a pretense for the three men in her life: brother, father,lover.

	There is no definite setting: The poem can happen anywhere, at any time as long as a woman finds a need to put a pretense to please the people she loves most.

	Yes. There are two conflicts in the poem: The original conflict, arising from the fact that she is not satisfied with her relationship with the three men. The next conflict is when these men sense that there is something more going on, that this woman is sitting behind a mask, and decided not to act upon it for fear of either losing what they ``love'' from her the most.

	three main images pop up throughout in the poem, all related to her interactions with the men while wearing the mask:\\
	1. We see a woman who is very different from her brother, as if opposites.

	2. We see a very obedient and chaste daughter.

	3. And lastly, we see a woman determined to please her lover.

	There is no meter, nor a rhyme: this poem was written in free verse.

	There are a couple of allusions in the poem: 
	
	1.King Lear is the name of a Shakespearean play.\\
	\\2.Billiken is a charm doll with pointed ears and a mischievous smile.\\
	\\3.Philomela is the name of a woman in Greek Mythology. She was raped. And her tongue cut out.\\

	The author takes a jaded tone throughout her poem: she is tired of the constant displeasure and inacceptance that these men find from her, and wants to change this once and for all by wearing a mask and adapting a different personality that, too her, will finally please that person.

	Summary: basically, this is a poem of a woman who wears wants to wear three different masks a.k.a. different personalities for the three most important men in her life: The father, the brother, and the lover. For the first mask, she wants to be the complete opposite of her brother. For the father, she wants to project the image of being the obedient daughter she used to be. For the lover, she wants to be all the lover loved from her and be loved by him again. In the end, each of these men catches a glimpse that what they're seeing a.k.a the masks she's wearing in front of them, isn't really the real her, but chose not to act upon it.

	The author is trying to tell her readers that despite seeing the need to wear such masks to please the people who are dearest to our heart, they can and will see through the farce that the masks create.




	\[The End\]

	
\end{document}

